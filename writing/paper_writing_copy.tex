\documentclass{article}

% these packages let you do math
\usepackage{amsmath}
\usepackage{amssymb}

% we need these packages for fancy R tables
\usepackage{booktabs}
\usepackage{float}
\usepackage{colortbl}
\usepackage{xcolor}

% these packages play with the spacing/margins of the document. Uncomment the commands on lines 16 and 17 to see what they do.
\usepackage{a4wide}
\usepackage{setspace}
\usepackage{geometry}
\usepackage{parskip}
%\doublespacing
%\geometry{margin=1.5in}

% this package helps us with including images. Setting the graphics path makes it easier to refer to things in the \includegraphics command.
\usepackage{graphicx}
\graphicspath{ {C:/Users/Haokun Zhang/Desktop/UTexas classes/casual inference/Hidden curriculum/NLSY97_Hidden_Curriculum/figures/} }

% make some hyperlinks using the \href command
\usepackage{hyperref}
\hypersetup{
    colorlinks=true,
    linkcolor=black,
    urlcolor=blue
}

% set the author, title, and date of the document. \maketitle adds it to the document.
\author{Haokun Zhang}
\title{Incarceration Status by Race and Gender in the Year 2002}
\date{Sping 2022}

\begin{document}
\maketitle

\section{Introduction}

In this paper, I summarized the incarceration status by gender and race in 2002. The analyze includes mean number of arrests by race and gender, as well as a regression of the effect of gender and race on incarceration numbers. The dataset is from \href{https://www.nlsinfo.org/investigator/pages/search}{NLS investigator}.

The formula of regression is listed below:

$$
    y = {race}\beta_1 +  {age\beta_2} + \varepsilon
$$

From Figure \ref{arrests_by_racegender} 

\begin{figure}[H]
    \begin{center}
        \includegraphics[width=.85\textwidth]{incarceration_by_racegender.png}
    \end{center}
    \caption{Mean Number of Incarceration in 2002 by Race and Gender}
    \label{arrests_by_racegender}
\end{figure}

\input{C:/Users/Haokun Zhang/Desktop/UTexas classes/casual inference/Hidden curriculum/NLSY97_Hidden_Curriculum/tables/incarceration_by_racegender.tex}

Again, the \texttt{equation} environment is preferred because the begin and end delimiters are different.

I can also add a bibliography, but this is beyond the scope of our discussion right now. Overleaf has plenty of resources for this on their \href{https://www.overleaf.com/learn}{website}. Another good place to look for LaTeX help is the \href{https://en.wikibooks.org/wiki/LaTeX}{WikiBook} on it.

\newpage

\section{The Second Section}

Wherein we do tables and graphs. To include the graph we made in ggplot, we create the \texttt{figure} environment. The `H' option tells LaTeX to `hold' the position of the figure instead of positioning it somewhere else. I use the \texttt{caption} command to add a caption{\textemdash}although I also put a title on the plot in ggplot so you would typically choose one or the other. I use the \texttt{label} command after the caption to add a label. Then in my paper I can use the \texttt{ref} command and LaTeX knows I am referring to Figure \ref{arrests_by_racegender}.

Tables are somewhat easier, since \texttt{kableExtra} and \texttt{stargazer} generate LaTeX code that is ready to just ``copy-paste'' into our document. The \texttt{label} argument in the R code is the label that the table will have in the tex output, if you want to \texttt{ref} it.

\input{C:/Users/Haokun Zhang/Desktop/UTexas classes/casual inference/Hidden curriculum/NLSY97_Hidden_Curriculum/tables/incarceration_by_racegender.tex}

\input{C:/Users/Haokun Zhang/Desktop/UTexas classes/casual inference/Hidden curriculum/NLSY97_Hidden_Curriculum/tables/regress_incarceration_by_racegender.tex}

\end{document}